%%
%% xkeyval-use-package.tex
%%
%% Xkeyval in package test file, to demonstrate xkeyval
%%
%% (c)2021, Jesse op den Brouw <J.E.J.opdenBrouw@hhs.nl>
%%
%% License: LPPL, https://www.latex-project.org/lppl/

\documentclass[12pt]{article}

\usepackage[english]{babel}
\usepackage{blindtext}
\usepackage{charter}
\usepackage{nimbusmono}
\usepackage[margin=2.2cm]{geometry}
\usepackage{parskip}

\usepackage{listings}
\lstset{
    language=[AlLaTeX]TeX,
    basicstyle=\footnotesize\ttfamily,
    backgroundcolor=\color{gray!10!white},
    moretexcs={setlength,or},
    escapeinside={(*}{*)}
}

% Set package options: parindent is 20 pt and text color is blue and style is small caps
\usepackage[parindent=25pt,color=blue,style=sc,showindent=true]{xkeyval-package}

\title{Example of the \texttt{xkeyval} package for package writers}
\author{J. op den Brouw\thanks{The Hague University of Applied Sciences, \texttt{J.E.J.opdenBrouw@hhs.nl}}}
\date{\today}

\begin{document}

\maketitle

This simple test package is conceived to demonstrate the use of \emph{key}/\emph{value} option processing for packages and macros. The test package can be loaded with:

\begin{lstlisting}
\usepackage[(*$\langle$\rmfamily\emph{options}$\rangle$*)]{xkeyval-package}
\end{lstlisting}

where \emph{options} is a list of options:

\texttt{parindent}=\emph{skip}, default 0pt\\
\texttt{color}=\emph{color}, default is black\\
\texttt{style}=\emph{up|sl|sc|it}, default is up\\
\texttt{showindent}=\emph{true|false}, default is false

As an example:

\begin{lstlisting}
\usepackage[parindent=15pt,color=blue,style=sc,showindent]{xkeyval-package}
\end{lstlisting}

The package has one macro:

\begin{lstlisting}
\mybox[(*$\langle$\rmfamily\emph{options}$\rangle$*)]{(* \rmfamily\emph{text} *)}
\end{lstlisting}

that typesets some simple text. The options are the same as the package options. The package options serve as the default for the macro, but can be overridden with per-macro invocations. As an example:
\begin{lstlisting}
\mybox[parindent=15pt,style=sc,color=blue,showindent=false]{\blindtext}
\end{lstlisting}

Best is to examine the package file in detail.

\newpage

\color{orange!80!black}
Some text before $\ldots$ \texttt{parindent} = \the\parindent

% Output text with all package options
\mybox{\blindtext}

% Output text with all macro options
\mybox[parindent=100pt,color=red,style=sl]{\blindtext}

% Output text with package parindent and greenish macro option
\mybox[color=green!20!black]{\blindtext}

% Output text with all package options
\mybox[showindent=false]{\blindtext}

Some text after $\ldots$ \texttt{parindent} = \the\parindent

\newpage
\color{black}

\lstinputlisting{xkeyval-package.sty}

\end{document}