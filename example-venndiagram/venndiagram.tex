

\documentclass[a4paper]{article}

\usepackage{venndiagram}
\usepackage{parskip}
\usepackage[left=1in,right=1in,showframe]{geometry}

\begin{document}

\section{Venn diagrams}
Venn diagrams can be used for describing logical functions. It consists of a
square or a rectangle which is called the \textsl{universe}. In the universe
we draw circles that represent the variables. Within a circle we say that the
variable is 1. Outside the circle a variable is said to be~0. The circles may
overlap; in that area it is said that both variables are 1. The universe
contains all possible sets and functions that can be made with the variables
and has the constant value of~1. To indicate the display of a function, we
shade a part of the Venn diagram that is needed. Figure~\ref{fig:vdaorb} shows
the Venn diagram representing the logic function $A+B$.

\begin{figure}[!ht]
\centering
\begin{venndiagram2sets}[radius=1.0cm]
\fillA \fillB
\end{venndiagram2sets}
\caption{Venn diagram for $A+B$.}
\label{fig:vdaorb}
\end{figure}

Similar, Figure~\ref{fig:vdaandb} shows the logic function $A\cdot B$.
One can easily see that $A\cdot B$ is contained in $A+B$.
\begin{figure}[!ht]
\centering
\begin{venndiagram2sets}[radius=1.0cm]
\fillACapB
\end{venndiagram2sets}
\caption{Venn diagram for $A\cdot B$.}
\label{fig:vdaandb}
\end{figure}

Figure~\ref{fig;vdaorborc} shows the Venn diagram for
the logic functions $A+B+C$.

\begin{figure}[!ht]
\centering
\begin{venndiagram3sets}[radius=1.0cm]
\fillA \fillB \fillC
\end{venndiagram3sets}
\caption{Venn diagram for $A+B+C$.}
\label{fig;vdaorborc}
\end{figure}

\newpage
Figure~\ref{fig:vddemorgan} shows the proof of one of DeMorgan's theorems.
\begin{figure}[!ht]
\centering
\begin{venndiagram2sets}[radius=1.0cm]
\fillNotAorB
\end{venndiagram2sets}
\caption{Venn diagram for $\overline{A+B}$ which is equal to $\overline{A}\cdot\overline{B}$.}
\label{fig:vddemorgan}
\end{figure}

Figure~\ref{fig:vdintersectionthree} shows the intersection of three variables.
\begin{figure}[!ht]
\centering
\begin{venndiagram3sets}[radius=1.0cm,labelA=$a$,labelB=$b$,labelC=$c$]
\fillACapBCapC
\end{venndiagram3sets}
\caption{Venn diagram for $a\cdot b\cdot c$.}
\label{fig:vdintersectionthree}
\end{figure}

Figure~\ref{fig:vdunion} shows the proof of one of the absorption laws.
\begin{figure}[!ht]
\centering
\begin{venndiagram2sets}[radius=1.0cm,labelA=$l$,labelB=$b$]
\fillA
\end{venndiagram2sets}\hspace*{0.5cm}%
\begin{venndiagram2sets}[radius=1.0cm,labelA=$l$,labelB=$b$]
\fillBNotA
\end{venndiagram2sets}\hspace*{0.5cm}%
\begin{venndiagram2sets}[radius=1.0cm,labelA=$l$,labelB=$b$]
\fillA\fillB
\end{venndiagram2sets}
\caption{Union of the logic function $l\vee(\neg l \wedge b)$.}
\label{fig:vdunion}
\end{figure}



\end{document}