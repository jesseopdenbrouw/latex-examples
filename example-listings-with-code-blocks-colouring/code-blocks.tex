% !TeX encoding = UTF-8
% !TeX spellcheck = nl_NL
% !TeX TS-program = xelatex

%% code-blocks.tex
%%
%% Listings with Code::Blocks coloring
%%
%% (c)2021, Jesse op den Brouw <J.E.J.opdenBrouw@hhs.nl>
%%
%% License: LPPL, https://www.latex-project.org/lppl/

\documentclass[a4paper,10pt]{article}


\usepackage{parskip}
\usepackage{etoolbox}

%% Find out what engine we're running...
\usepackage{ifluatex,ifxetex}

%% Different setup for lua and xe
\ifnum 0\ifxetex 1\fi\ifluatex 1\fi>0
\usepackage[math-style=TeX]{unicode-math}
\usepackage{fontspec}
\setmainfont[Ligatures=TeX]{Calibri}
\defaultfontfeatures{Scale=MatchUppercase}
\setsansfont{Calibri}
\setmonofont{Consolas}
\setmathfont[slash-delimiter=frac]{Cambria Math}
\setmathfont[range=up]{Calibri}
\setmathfont[range=it]{Calibri Italic}
\setmathfont[range=bfup]{Calibri Bold}
\setmathfont[range=bfit]{Calibri Bold Italic}
\setoperatorfont\normalfont % For log, sin, cos, etc.
\else
\usepackage[utf8]{inputenc}
\usepackage{nimbusmono}
\usepackage[scaled]{helvet}
\let\oldhrulefill\hrulefill
\usepackage[charter]{mathdesign}
\let\hrulefill\oldhrulefill
\fi

%% Use computer code listings in the style of Code::Blocks
\usepackage{listings}
\makeatletter
%% ) is now a token
\patchcmd{\lsthk@SelectCharTable}{`)}{``}{}{}
%% Replace \char32 with U-form vrule/hrule to match the tab char
\AtBeginDocument{%
\def\lst@visiblespace{\kern.06em\hbox{\vrule\@height.3ex}%
                      \hrulefill\hbox{\vrule\@height.3ex}}}
\makeatother

%% Use textcomp for upright quotes in listings
\usepackage{textcomp}
\usepackage{xcolor}
\definecolor{dkgreen}{rgb}{0,0.6,0}
\definecolor{gray}{rgb}{0.5,0.5,0.5}
\definecolor{mauve}{rgb}{0.58,0,0.82}
\definecolor{lightgray}{rgb}{0.95,0.95,0.95}
\definecolor{GKYgray}{rgb}{0.65,0.65,0.65}
\definecolor{thuasgreen}{RGB}{158,167,0}
\definecolor{lightpink}{HTML}{F8E0E0}%{FBF3F3}%
\definecolor{cppgreen}{RGB}{0,160,0}
\definecolor{comment}{RGB}{128,128,255}
\definecolor{keyword}{RGB}{0,0,160}
\definecolor{number}{RGB}{240,0,240}
\definecolor{character}{RGB}{224,160,0}

\newcommand{\CodeSymbol}[1]{\textcolor{red}{#1}}
%% Define listing style for VHDL
\lstset{%
  language=C,
  basicstyle=\small\ttfamily,
  numbers=left,
  numberstyle=\tiny\color{gray},
  stepnumber=1,                           
  numbersep=8pt,
  keywordstyle=\bfseries\color{keyword},
  stringstyle=\color{blue},
  commentstyle=\itshape\color{comment},
  morecomment=[l][\color{cppgreen}]{\#},
  showspaces=false,
  showstringspaces=false,
  showtabs=false,
  frame=lines,
  rulecolor=\color{black},
  tabsize=4,
  captionpos=b,
  breaklines=true,
  breakatwhitespace=false,
  title=\lstname,
  postbreak=\mbox{\textcolor{red}{$\hookrightarrow$}\space},
  upquote=true,
  aboveskip=1ex,
  belowskip=2ex,
  escapeinside={(*}{*)},
  literate=%
    *{0}{{{\textcolor{number}0}}}1
    {1}{{{\textcolor{number}1}}}1
    {2}{{{\textcolor{number}2}}}1
    {3}{{{\textcolor{number}3}}}1
    {4}{{{\textcolor{number}4}}}1
    {5}{{{\textcolor{number}5}}}1
    {6}{{{\textcolor{number}6}}}1
    {7}{{{\textcolor{number}7}}}1
    {8}{{{\textcolor{number}8}}}1
    {9}{{{\textcolor{number}9}}}1
    {.0}{{{\textcolor{number}{.0}}}}2
    {.1}{{{\textcolor{number}{.1}}}}2
    {.2}{{{\textcolor{number}{.2}}}}2
    {.3}{{{\textcolor{number}{.3}}}}2
    {.4}{{{\textcolor{number}{.4}}}}2
    {.5}{{{\textcolor{number}{.5}}}}2
    {.6}{{{\textcolor{number}{.6}}}}2
    {.7}{{{\textcolor{number}{.7}}}}2
    {.8}{{{\textcolor{number}{.8}}}}2
    {.9}{{{\textcolor{number}{.9}}}}2
    {.e}{{{\textcolor{number}{.e}}}}2
    {.E}{{{\textcolor{number}{.E}}}}2
    {0e}{{{\textcolor{number}{0e}}}}2
    {1e}{{{\textcolor{number}{1e}}}}2
    {2e}{{{\textcolor{number}{2e}}}}2
    {3e}{{{\textcolor{number}{3e}}}}2
    {4e}{{{\textcolor{number}{4e}}}}2
    {5e}{{{\textcolor{number}{5e}}}}2
    {6e}{{{\textcolor{number}{6e}}}}2
    {7e}{{{\textcolor{number}{7e}}}}2
    {8e}{{{\textcolor{number}{8e}}}}2
    {9e}{{{\textcolor{number}{9e}}}}2
    {0E}{{{\textcolor{number}{0e}}}}2
    {1E}{{{\textcolor{number}{1E}}}}2
    {2E}{{{\textcolor{number}{2E}}}}2
    {3E}{{{\textcolor{number}{3E}}}}2
    {4E}{{{\textcolor{number}{4E}}}}2
    {5E}{{{\textcolor{number}{5E}}}}2
    {6E}{{{\textcolor{number}{6E}}}}2
    {7E}{{{\textcolor{number}{7E}}}}2
    {8E}{{{\textcolor{number}{8E}}}}2
    {9E}{{{\textcolor{number}{9E}}}}2
    {\{}{{\CodeSymbol{\{}}}1
    {\}}{{\CodeSymbol{\}}}}1
    {(}{{\CodeSymbol{(}}}1
    {)}{{\CodeSymbol{)}}}1
    {[}{{\CodeSymbol{[}}}1
    {]}{{\CodeSymbol{]}}}1
    {>}{{\CodeSymbol{$>$}}}1
    {<}{{\CodeSymbol{$<$}}}1
    {=}{{\CodeSymbol{$=$}}}1
    {;}{{\CodeSymbol{$;$}}}1
    {,}{{\CodeSymbol{$,$}}}1
    {*}{{\CodeSymbol{$*$}}}1
    {+}{{\CodeSymbol{$+$}}}1
    {-}{{\CodeSymbol{$-$}}}1
    {!}{{\CodeSymbol{!}}}1
    {\&}{{\CodeSymbol{\&}}}1,
    moredelim=[s][\textcolor{character}]{'}{'}
}



\begin{document}

In this document, you will find a few C program listings rendered with the color scheme used in the Code::Blocks IDE.
\vspace*{1cm}

\begin{lstlisting}
#include <stdio.h>
#include <math.h>

int main(void) {
    double x, y, wortel, macht;
    int i;
    printf("Voer een positief getal x in: ");
    scanf("%lf", &x);
    printf("Voer een getal y in: ");
    scanf("%lf", &y);
    for (i = 0; i < 5; i = i + 1) {
        wortel = sqrt(x);
        macht = pow(x, y);
        printf("De vierkantswortel uit %f is: %f\n", x, wortel);
        printf("%f tot de macht %f is: %f\n", x, y, macht);
        x = x + 1;
    }
    fflush(stdin);
    getchar();
    return 0;
}
\end{lstlisting}

\newpage

\begin{lstlisting}
#include <stdio.h>

int main(void){
    int cijfer;
    char letter;
    do {
        printf("Geef je cijfer: ");
        scanf("%d", &cijfer);
    } while (cijfer < 0 || cijfer > 10);
    switch (cijfer) {
        case 10:
        case 9:
        case 8:
            letter = 'A'; break;
        case 7: 
            letter = 'B'; break;
        case 6: 
            letter = 'C'; break;
        case 5:
            letter = 'D'; break;
        default:
            letter = 'F'; break;
    }
    printf("Dit komt overeen met een %c.\n", letter);
    fflush(stdin);
    getchar();
    return 0;
}
\end{lstlisting}

\newpage

\begin{lstlisting}
#include <stdio.h>

int fib(int n) {
    if (n < 2) {
        return n;
	}
	else {
	  return fib(n-1) + fib(n-2);
	}
} 

int main(void) {
    int getal;

    do {
        printf("Geef een getal groter dan of gelijk aan 0: ");
        scanf("%d", &getal);
    } while (getal < 0);

    printf("fib(%d) is %d\n", getal, fib(getal));

    printf("Druk op enter om het programma af te sluiten.");
    fflush(stdin);
    getchar();
    return 0;
}
\end{lstlisting}

\newpage

\begin{lstlisting}
#include <stdio.h>

/* Dit programma demonstreert hoe een
   groot getal cijfer voor cijfer kan
   worden ingelezen */

int leesCijfer(void) {
    switch (getchar()) {
        case '0': return 0;
        case '1': return 1;
        case '2': return 2;
        case '3': return 3;
        case '4': return 4;
        case '5': return 5;
        case '6': return 6;
        case '7': return 7;
        case '8': return 8;
        case '9': return 9;
        default: return -1;
    }
}

int main(void) {
    int cijfer;
    printf("Type een groot getal:\n");
    do {
        cijfer = leesCijfer();
        if (cijfer != -1) {
            printf("%d ", cijfer);
        }
    } while (cijfer != -1);
    
    return 0;
}
\end{lstlisting}

\newpage

\begin{lstlisting}
#include <stdio.h>

#define AANTAL 10

int main(void) {
    double temperatuur[AANTAL];
    int i;
    double temp_acc;

    for (i = 0; i < AANTAL; i = i + 1) {
        do {
            printf("Geef temperatuur %d op: ", i + 1);
            fflush(stdin);
        } while (scanf("%lf", &temperatuur[i]) != 1);
    }

    temp_acc = 0.0;
    for (i = 0; i < AANTAL; i = i + 1) {
        temp_acc = temp_acc + temperatuur[i];
    }

    printf("De gemiddelde temperatuur is %.2f\n\n", temp_acc / AANTAL);

    printf("Druk op enter om het programma af te sluiten.");
    fflush(stdin);
    getchar();
    return 0;
}
\end{lstlisting}

\newpage

\begin{lstlisting}
#include <stdio.h>
#include <malloc.h>

int main(void) {

	char *ptr = malloc(1024);  /* Allocate 1024 bytes */

	if (ptr == NULL) {
		printf("Oops, no memory!\n");
		exit(-2);
	}

	printf("Allocated 1024 bytes at address %p\n", ptr);

	free(ptr); /* Free memory */

	return 0;
}
\end{lstlisting}


\begin{lstlisting}
	char *ptrnew;

	ptrnew = realloc(ptr, 2048);
\end{lstlisting}

\newpage

\begin{lstlisting}
/* myprog.c */
#include <stdio.h>
#include <stdlib.h>

int main(int argc, char *argv[]) {

    int i;

    printf("\nAantal argumenten: %d\n\n", argc);
    for (i = 0; i<argc; i++) {
        printf("Argument %d: %s\n", i, argv[i]);
    }
    return 0;
}
\end{lstlisting}

\end{document}