\documentclass{article}

\usepackage{mathtools}
\usepackage{tikz}
\usepackage{tikz-3dplot}
\usepackage[charter]{mathdesign}
\usepackage{parskip}

\begin{document}

\begin{figure}[!ht]
% set the viewing angle
\def\a{45} % DO NOT CHANGE
\tdplotsetmaincoords{65}{\a}
\centering
\begin{tikzpicture}
	[scale=7,
	tdplot_main_coords,
	axis/.style={->,black,very thin},
	curve/.style={black,thin},rotate=95]
	% radius and axes
	\def\radius{.1}
	\def\axissize{.3}
    % length
	\def\th{1} % DO NOT CHANGE

	% intervals on the z-axis
%	\draw[axis,-,dashed,red] (0,0,0) -- (0,0,\th);
	\draw[axis,thick] (0,0,\th-0.5*\axissize) -- (0,0,\th+0.5*\axissize) node[anchor=south]{$I$};
	\draw[axis,thick] (0,0,-.8+\th+0.25*\axissize) -- (0,0,-.8+\th-0.25*\axissize) node[anchor=north,yshift=-0.5mm]{$I_e$};
     
     
	\tdplotsinandcos{\sintheta}{\costheta}{0}
	% upper end
	\tdplotdrawarc[curve,thick]{(0,0,\th)}{\radius*\costheta}{\a-360}{\a}{}{}
	% single sections
	\foreach \height in {0.5}{
    	% draw the front half with a solid line
		\tdplotdrawarc[curve,solid]{(0,0,\height)}{\radius*\costheta}{\a-180}{\a}{}{}
        % and the back half with a dotted line
        \tdplotdrawarc[curve,dotted]{(0,0,\height)}{\radius*\costheta}{\a-360}{\a-180}{}{}
		\tdplotdrawarc[curve,thick,fill=brown!50!red,opacity=0.5]{(0,0,\height)}{\radius*\costheta}{\a-360}{\a}{}{}
    }
	% lower end
	\tdplotdrawarc[curve,thick,dashed]{(0,0,0)}{\radius*\costheta}{\a}{\a+180}{}{}
	\tdplotdrawarc[curve,thick]{(0,0,0)}{\radius*\costheta}{\a}{\a-180}{}{}
	% right line and labels
	\tdplotsinandcos{\sintheta}{\costheta}{\a}
	\draw[thick] (\radius*\costheta,\radius*\sintheta,0) {} -- (\radius*\costheta,\radius*\sintheta,\th);
	\draw[latex-latex] (1.4*\radius*\costheta,\radius*\sintheta,0) -- node[above] {$L$} (\radius*\costheta,1.4*\radius*\sintheta,\th);
	% left line
	\tdplotsinandcos{\sintheta}{\costheta}{\a+180}
	\draw[thick] (\radius*\costheta,\radius*\sintheta,0) -- (\radius*\costheta,\radius*\sintheta,\th);
	
	% label of the diameter
	\draw[latex-latex] (\radius,0,\th/2) -- (-\radius,0,\th/2) node[anchor=south west,yshift=1mm,xshift=0.6mm] {$d$};
    % Draw the electrons
	\foreach \ang/\dis in {0/0,1/.06,-1/-0.04,-0.5/-0.1,-0.6/-0.7,-3/-.8,1/-.8} {
		\shade[ball color=brown!50!red,opacity=0.90] (-\a+\ang:-1.5-\dis) circle (0.008cm) node[yshift=-1.5mm,font=\tiny] {\textsl{q}\raisebox{.7ex}{-}};
	}	
\end{tikzpicture}
\caption{A section of a metal wire.}
\end{figure}

When applying a \emph{voltage} across a metal wire, a \emph{current} (designated as $I$) will flow. A voltage can be seen as electric pressure between two points. A current can be seen as movement of electric particles in the wire, in this case electrons. Note that the electrons, having a negative charge, move in the opposite direction of the measured current (as shown by $I_e$).

The resistance of a metal wire with length $L$ and diameter $d$ is:

\begin{equation}
R = \frac{\rho L}{A}
\end{equation}

where $A$ is the area of the cross section with

\begin{equation}
A = \dfrac{1}{4}\pi d^2
\end{equation}

and $\rho$ is the specific specific resistance of the metal in $\Omega$ m (ohm meter).

\end{document}

% Thanks to Jörg Petzold for creating this example
