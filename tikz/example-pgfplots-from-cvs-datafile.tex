\documentclass[a4paper]{article}

\usepackage{pgfplots}
\usepackage{siunitx}
\usepackage[margin=1in]{geometry}
\usepackage{parskip}

\author{Jesse E.\@ J.\@ op den Brouw\thanks{\sffamily J.E.J.opdenBrouw@hhs.nl}}
\title{Some graphs of an NTC resistor}
\date{\today}

\begin{document}
\maketitle
\vspace*{2cm}
\begin{abstract}
This document shows some interesting graph of the parameters of the Betatherm 10K3A542I NTC thermistor.
See {\sffamily http://www.farnell.com/datasheets/69441.pdf?\_ga=1.32882430.1374476496.1461826737}
\end{abstract}
\newpage

The graph in Figure~\ref{fig1} shows the resistance of the NTC versus the temperature. Note that the temperature is in degrees Celsius. Also note that the resistance is heavy non-linear.

\begin{figure}[!ht]
\begin{tikzpicture}
\begin{axis}[
    width=15cm,
    height=7cm,
    title=Resistance vs.\@ temparature.,
    xlabel=Temperature / \si{\degreeCelsius},
    ylabel=Resistance / \si{\ohm},
    skip coords between index={0}{40}
    ]
\addplot table [x index={0},y index={1}, mark=none] {GegevensBetatherm10K3A542I.dat};
\end{axis}
\end{tikzpicture}
\caption{Resistance versus temperature of an NTC resistor.}
\label{fig1}
\end{figure}

The graph in Figure~\ref{fig2} shows the \emph{natural logarithm} of the resistance versus the \emph{inverted} temperature, so the following formula applies: $\ln R = f(1/T)$. Note that this graph shows an \emph{almost} straight line

\begin{figure}[!ht]
\begin{tikzpicture}
\begin{axis}[
    width=15cm,
    height=7cm,
    title=Natural logarithm of resistance vs.\@ inverted temparature.,
    xlabel=Inverted temperature / \si{\per\kelvin},
    ylabel=Nat.\@ log.\@ Resistance / $\ln$ \si{\ohm}
    ]
    \addplot table [x index={2},y index={3}, mark=none] {GegevensBetatherm10K3A542I.dat};
\end{axis}
\end{tikzpicture}
\caption{Natural logarithm of the resistance versus the inverted temperature of an NTC resistor.}
\label{fig2}
\end{figure}

An important parameter of an NTC resistor is the beta coefficient. One take this coefficient as a constant, but the graph in Figure~\ref{fig3} clearly shows that this is \emph{not} the case.

\begin{figure}[!ht]
\begin{tikzpicture}
\begin{axis}[
    width=15cm,
    height=7cm,
    title=Beta coefficient versus temperature.,
    xlabel=Temparature / \si{\degreeCelsius}
    ylabel=Beta coeffient / \si{\kelvin},
    ]
    \addplot table [x index={0},y index={6}, mark=none, smooth] {GegevensBetatherm10K3A542I.dat};
\end{axis}
\end{tikzpicture}
\caption{Beta coefficient versus the temperature of an NTC resistor.}
\label{fig3}
\end{figure}

\end{document}