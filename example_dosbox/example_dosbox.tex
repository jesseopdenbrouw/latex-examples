\documentclass[12pt]{article}

\usepackage[left=1in,right=1in,top=1in,bottom=1.5in,footskip=0.4in]{geometry}
\usepackage{caption}
\usepackage{parskip}
\usepackage{charter}
\usepackage{nimbusmono}

%% Use computer code listings
\usepackage{listings}
%% Use textcomp for upright quotes in listings
\usepackage{textcomp}

\lstdefinestyle{lstempty}{
  basicstyle=\small\ttfamily,
  numbers=none,
  backgroundcolor=\color{white},
  showspaces=false,
  breaklines=true,
  breakatwhitespace=false,
  frame=none,
  title={},
  upquote=true,
  aboveskip=0.5\baselineskip,
  belowskip=-0\baselineskip,
  escapeinside={(*}{*)},
}

\usepackage[most]{tcolorbox}

\AtBeginDocument{%
\newtcblisting[blend into=figures]{dosbox}[1][title=please fill in]{%
  before upper={\tcbsubtitle[colback=white,toprule=0mm]{\includegraphics{commandprompt.png}\hspace*{0.5em}\footnotesize\sffamily \raisebox{0.2ex}{Command Prompt}}},
  top=0pt,
  bottomtitle=3pt,
  colback=white,
  colframe=black,
  coltitle=black,
  enhanced,
  listing only,
  sharp corners,
  listing remove caption=true,
  listing options={style=lstempty},
  attach boxed title to bottom center={yshift=-10pt},
  boxed title style={enhanced jigsaw, colback=white, boxrule=0pt,},
  fonttitle=\footnotesize,
  parbox=false,
  floatplacement=!ht,
  float=!ht,
  left=7pt,
  right=3pt,
  %blend before title code={\bfseries Figure \thefigure:~\normalfont\slshape},
  #1
}
} % AtBeginDocument

\begin{document}

This small example shows how to simulate a DOS-box as frequently seen in Windows programs.
It makes use of the \texttt{tcolorbox} and \texttt{listings} packages.

For output of the program, see Figure~\ref{fig:output}.

\begin{dosbox}[title=Output of some program.,label=fig:output]
Output of some program.
Input by user: (*\textbf{123}*)
Input by user (*\underline{123}*)
Input by user (*\emph{123}*)
Output of program (*\hfill\normalfont\emph{(Some text on the right)}*)
\end{dosbox}

Use \texttt{title=...} to set your caption text.

Use \texttt{blend before title code} to explicitly copy the format of your caption setup.

\end{document}
