%
% This LaTeX document calculates the first n Fibonacci numbers
% for 2 < n < 48. For n > 47, overflow occurs.
%
% (c)2018, Jesse op den Brouw <J.E.J.opdenBrouw@hhs.nl>
%
% License: LPPL, https://www.latex-project.org/lppl/

\documentclass[a4paper]{article}

\begin{document}
\raggedright

% The counters
\newcount\n \newcount\fibo \newcount\fiboprev \newcount\temp

% Print Fibonacci numbers using an iterative approach
\def\fibonacci#1{%
\n=#1 \advance\n by -2 \fibo=1 \fiboprev=0
0, 1%
\loop
	\temp=\fibo
	\advance\fibo by \fiboprev
	\fiboprev=\temp
	\ifnum\n>1 , \else{} en~\fi
	\the\fibo
	\advance\n by -1\relax
	\ifnum\n>0
\repeat
}

\noindent
De eerste 47 Fibonacci-nummers zijn \fibonacci{47}.

\end{document}
