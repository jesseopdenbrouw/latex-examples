\documentclass[12pt,fleqn]{article}

\usepackage[english]{babel}
\usepackage[a4paper,left=1in,right=1in,top=1in,bottom=1.5in]{geometry}
\usepackage{parskip}
\usepackage{mathtools}
\setlength{\mathindent}{1em}
\usepackage{lmodern}
\usepackage{xcolor}
\usepackage{tikz}
\usetikzlibrary{decorations.pathreplacing}

%%
%% based on https://tex.stackexchange.com/questions/476045/euler-and-minus-sign
%%
%% Imaginary unit, e, and Euler
\newcommand\imaginaryunit{j}                   % the imaginary unit, i for mathematician and theoretical physicist,
                                               % j for the rest of the world.
\newcommand\imunit{\mathrm{\imaginaryunit}}    % ... in upright math
\newcommand\ce{\mathrm{e}}                     % the constant e, upright of course

\newcommand{\epowre}[1]{\ce^{#1}}              % e to the power of real
\makeatletter
\newcommand{\fiximunit@@}{\if\imaginaryunit j\,\fi}
\newcommand{\epowim}[1]{\ce^{\epowim@#1}}      % e to the power of imaginary
\newcommand{\epowim@}{\@ifnextchar-{\epowim@@}{\epowim@@{\fiximunit@@}}}
\newcommand{\epowim@@}[1]{#1\imunit}
\newcommand{\epowim@@@}{\@ifnextchar-{\epowim@@@@}{+\epowim@@@@{}}}
\newcommand{\epowim@@@@}[1]{#1\imunit}
\newcommand{\epowcom}[2]{\ce^{#1\epowim@@@#2}} % e to the power of complex
\newcommand{\cis}[1]{\cis@#1}                  % cos + imaginary sin
\newcommand{\cis@}{\@ifnextchar-{\cis@@@}{\cis@@}}
\newcommand{\cis@@}[1]{\cos#1 + \imunit\sin#1}
\newcommand{\cis@@@}[2]{\cos#2 - \imunit\sin#2}
\makeatother
%\renewcommand{\Re}{\mathrm{Re}}  % Redefine \Re
%\renewcommand{\Im}{\mathrm{Im}}  % Redefine \Im




\begin{document}
\section*{Macros for using complex numbers with \imaginaryunit}
The imaginary unit is: \imaginaryunit.

Set the imaginary unit with: \verb|\renewcommand{\imaginaryunit}{j}| (default)

In math-mode: $\imunit$. Use in math-mode: \verb|\imunit|

The constant e in math-mode $\ce$. Use in math-mode: \verb|\ce|

Real power of e: $\epowre{-2} = 0,13533528\ldots$ Use in math-mode: \verb|\epowre{arg}|

Imaginary power of e: $\epowim{\alpha}$. Use in math-mode: \verb|\epowim{arg}|

Goniometric complex: $\cis{\alpha}$. Use in math-mode: \verb|\cis{\alpha}|

Goniometric complex: $\cis{-\alpha}$. Use in math-mode: \verb|\cis{-\alpha}|

Goniometric complex: $\cis{{\omega t}}$. Use in math-mode: \verb|\cis{\omega t}}|

Goniometric complex: $\cis{-{\omega t}}$. Use in math-mode: \verb|\cis{-{\omega t}}|

Goniometric complex: $\cis{{-\omega t}}$. Use in math-mode: \verb|\cis{{-\omega t}}|

Complex power of e: $\epowcom{\sigma}{\omega t}$. Use in math-mode: \verb|\epowcom{\sigma}{\omega t}|

Complex power of e: $\epowcom{\sigma}{-\omega t}$. Use in math-mode: \verb|\epowcom{\sigma}{-\omega t}|

Together: $\epowcom{\sigma}{\omega t} = \epowre{\sigma}(\cis{{\omega t}})$

Together: $\epowcom{\sigma}{-\omega t} = \epowre{\sigma}(\cis{-{\omega t}})$


\begin{figure}[!h]
\centering
\begin{tikzpicture}[scale=2.5,font=\footnotesize] % scale is radius in cm
\def\uangle{40}
\draw[thick] (0,0) circle (1cm);
\draw[-latex,thick] (0,-1.1) -- (0,1.1) node[above] {$\Im$};
\draw[-latex,thick] (-1.1,0) -- (1.1,0) node[right] {$\Re$};
\draw[-latex,thick] (0,0) --(\uangle:1) node[midway,above] {1};
\draw[dashed] (\uangle:1) -| (0,0);
\draw[dashed] (\uangle:1) |- (0,0);
\draw[decorate,decoration={brace,amplitude=4pt,mirror,raise=2pt}] (0,0) -- node[below,yshift=-2mm]{$\cos\alpha$} ({cos(\uangle)},0);
\draw[decorate,decoration={brace,amplitude=4pt,raise=2pt}] (0,0) -- node[left,xshift=-2mm]{$\sin\alpha$} (0,{sin(\uangle)});
\draw[-latex] (0.3,0) arc (0:\uangle:0.3);
\node at (\uangle/2:0.4) {$\alpha$};
\node at (1,0) [xshift=50pt,right,font=\normalsize] {$\epowim{\alpha} = \cis{\alpha}$};
\end{tikzpicture}
\caption{Complex unit circle.}
\end{figure}

\newpage
\renewcommand{\imaginaryunit}{i}
\section*{Macros for using complex numbers with \imaginaryunit}
The imaginary unit is: \imaginaryunit.

Set the imaginary unit with: \verb|\renewcommand{\imaginaryunit}{i}|

In math-mode: $\imunit$. Use in math-mode: \verb|\imunit|

The constant e in math-mode $\ce$. Use in math-mode: \verb|\ce|

Real power of e: $\epowre{-2} = 0,13533528\ldots$ Use in math-mode: \verb|\epowre{arg}|

Imaginary power of e: $\epowim{\alpha}$. Use in math-mode: \verb|\epowim{arg}|

Goniometric complex: $\cis{\alpha}$. Use in math-mode: \verb|\cis{\alpha}|

Goniometric complex: $\cis{-\alpha}$. Use in math-mode: \verb|\cis{-\alpha}|

Goniometric complex: $\cis{{\omega t}}$. Use in math-mode: \verb|\cis{\omega t}}|

Goniometric complex: $\cis{-{\omega t}}$. Use in math-mode: \verb|\cis{-{\omega t}}|

Goniometric complex: $\cis{{-\omega t}}$. Use in math-mode: \verb|\cis{{-\omega t}}|

Complex power of e: $\epowcom{\sigma}{\omega t}$. Use in math-mode: \verb|\epowcom{\sigma}{\omega t}|

Complex power of e: $\epowcom{\sigma}{-\omega t}$. Use in math-mode: \verb|\epowcom{\sigma}{-\omega t}|

Together: $\epowcom{\sigma}{\omega t} = \epowre{\sigma}(\cis{{\omega t}})$

Together: $\epowcom{\sigma}{-\omega t} = \epowre{\sigma}(\cis{-{\omega t}})$

\renewcommand{\Re}{\mathrm{Re}}  % Redefine \Re
\renewcommand{\Im}{\mathrm{Im}}  % Redefine \Im

\begin{figure}[!h]
\centering
\begin{tikzpicture}[scale=2.5,font=\footnotesize] % scale is radius in cm
\def\uangle{-40}
\draw[thick] (0,0) circle (1cm);
\draw[-latex,thick] (0,-1.1) -- (0,1.1) node[above] {$\Im$};
\draw[-latex,thick] (-1.1,0) -- (1.1,0) node[right] {$\Re$};
\draw[-latex,thick] (0,0) --(\uangle:1) node[midway,below] {1};
\draw[dashed] (\uangle:1) -| (0,0);
\draw[dashed] (\uangle:1) |- (0,0);
\draw[decorate,decoration={brace,amplitude=4pt,raise=2pt}] (0,0) -- node[above,yshift=2mm]{$\cos\alpha$} ({cos(\uangle)},0);
\draw[decorate,decoration={brace,amplitude=4pt,mirror,raise=2pt}] (0,0) -- node[left,xshift=-2mm]{$\sin\alpha$} (0,{sin(\uangle)});
\draw[-latex] (0.3,0) arc (0:\uangle:0.3);
\node at (\uangle/2:0.4) {$\alpha$};
\node at (1,0) [xshift=50pt,right,font=\normalsize] {$\epowim{-\alpha} = \cis{-\alpha}$};
\end{tikzpicture}
\caption{Complex unit circle.}
\end{figure}

\end{document}

