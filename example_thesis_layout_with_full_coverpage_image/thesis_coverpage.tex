% !TeX TS-program = pdflatex

\documentclass[a4paper,12pt]{book} % Layout dubbelzijdig
\usepackage{tikz} % Nodig voor voorkant
\usepackage[outer=3.0cm,top=3cm,inner=2.0cm,bottom=3cm,bindingoffset=0.5cm,marginparwidth=5mm,marginparsep=5mm]{geometry} % Instellen pagina-opmaak
%\usepackage{charter} % Lettertype
%\usepackage[sfdefault,lining]{FiraSans} %% option 'sfdefault' activates Fira Sans as the default text font
%\renewcommand*\oldstylenums[1]{{\firaoldstyle #1}}
%\usepackage[T1]{fontenc}
\usepackage[T1]{fontenc}
\usepackage[sfdefault,lining]{FiraSans}
\usepackage{newtxsf}
\usepackage{kantlipsum} % Makkelijk proeftekst
\usepackage[dutch]{babel} % Nederlands als taal
\usepackage{parskip} % Niet inspringen en ruimte tussen alinea's
\usepackage[onehalfspacing]{setspace} % Regelafstand 1.5

%% Clear header and footer for blank pages, possibly print "This page ..."
%\newcommand{\thispageintentionallyleftblank}{Deze pagina is opzettelijk leeg gelaten}
\newcommand{\thispageintentionallyleftblank}{}
\makeatletter
\renewcommand*{\cleardoublepage}{\clearpage\if@twoside \ifodd\c@page\else
\hbox{}\vfill\begin{center}\thispageintentionallyleftblank\end{center}\vfill\vfill%
\thispagestyle{empty}%
\newpage%
\if@twocolumn\hbox{}\newpage\fi\fi\fi}
\makeatother

%% Create clickable (hyper)links
%% http://mirror.koddos.net/CTAN/macros/latex/contrib/hyperref/doc/manual.pdf
\usepackage{hyperref}
\hypersetup{
%	plainpages=false,
	breaklinks=true,
    colorlinks=true,
    linkcolor=black,
    linkcolor=black,
    citecolor=black,
    urlcolor=black,
    filecolor=black,
    pdfdisplaydoctitle=true,
    pdfstartview=FitH,
    pdfpagelayout=TwoPageRight,
}

%% Change appearance of titles
%% http://archive.cs.uu.nl/mirror/CTAN/macros/latex/contrib/titlesec/titlesec.pdf
%\RequirePackage{titlesec}
%\RequirePackage{titletoc}
%\titleformat{\chapter}[hang]{\Huge\bfseries}{\thechapter.}{0.7em}{}
%\titleformat{\section}{\large\bfseries}{\thesection}{1em}{}
%\titleformat{\subsection}{\bfseries}{\thesubsection}{1em}{}
%\newlength{\aftersubtitle}
%\setlength{\aftersubtitle}{1.2\baselineskip}
%\newlength{\aftersubsection}
%\setlength{\aftersubsection}{\aftersubtitle}
%\addtolength{\aftersubsection}{-\baselineskip}
%\titlespacing*{\section}{0pt}{\baselineskip}{\aftersubsection}
%\titlespacing*{\subsection}{0pt}{.8\baselineskip}{\aftersubsection}
%\titlespacing*{\subsubsection}{0pt}{.6\baselineskip}{0pt}

%% Using fancy headers and footers
%% http://ftp.snt.utwente.nl/pub/software/tex/macros/latex/contrib/fancyhdr/fancyhdr.pdf
\usepackage{fancyhdr}
\pagestyle{fancy}
\addtolength{\headwidth}{\marginparsep}
\addtolength{\headwidth}{\marginparwidth}
\renewcommand{\chaptermark}[1]{\markboth{#1}{}}
\renewcommand{\sectionmark}[1]{\markright{\thesection\ #1}}
\fancyhf{}
\fancyhead[LE,RO]{\small\textbf{\thepage}}
\fancyhead[LO]{\small\textbf{\rightmark}}
\fancyhead[RE]{\small\textbf{\leftmark}}
\fancypagestyle{plain}{%
\fancyhead{} % get rid of headers
\renewcommand{\headrulewidth}{0pt} % and the line
}

%% Making captions nicer...
%% http://ftp.snt.utwente.nl/pub/software/tex/macros/latex/contrib/caption/caption-eng.pdf
%% http://mirror.koddos.net/CTAN/macros/latex/contrib/caption/subcaption.pdf
\RequirePackage[font=footnotesize,format=plain,labelfont=bf,textfont=sl]{caption}
\RequirePackage[labelformat=simple,font=footnotesize,format=plain,labelfont=bf,textfont=sl]{subcaption}
\captionsetup[figure]{justification=centering,singlelinecheck=off,belowskip=-1ex}
\captionsetup[table]{justification=centering,singlelinecheck=off,skip=1ex}
\captionsetup[subfigure]{justification=centering,singlelinecheck=off,skip=3pt}
\captionsetup[subtable]{justification=centering,singlelinecheck=off,skip=3pt}
%% Put parens around the subfig name (a) (b) etc.
\renewcommand\thesubfigure{(\alph{subfigure})}
\renewcommand\thesubtable{(\alph{subtable})}

\def\coverpagefontstudent{\fontsize{30}{40}\selectfont\normalfont}
\def\coverpagefontopleiding{\fontsize{20}{30}\selectfont\normalfont}
\def\coverpagefonttitle{\fontsize{40}{50}\selectfont\bfseries}
\def\coverpagefontisubtitle{\fontsize{25}{35}\selectfont\normalfont}
\def\coverpagefontthuas{\fontsize{40}{50}\scshape\selectfont}
\definecolor{thuasgreen}{RGB}{158,167,0}
\definecolor{thuasgrey}{RGB}{34,51,67}

\begin{document}
  \thispagestyle{empty}
  \begin{tikzpicture}[remember picture,overlay]
      \node (image) [shape=rectangle,draw=none, minimum width=\paperwidth, minimum height=\paperheight] at (current page.center) {\includegraphics[width=\paperwidth,height=\paperheight]{photo.jpg}};

      \node (thuas) [shape=rectangle, fill=thuasgreen, minimum height=40mm, minimum width=\paperwidth, anchor=south west,opacity=0.5] at (current page.south west) {};
      \node[white] at (thuas.center) {\coverpagefontthuas De Haagse Hogeschool};

      \node (bar) [shape=rectangle, fill=thuasgrey, minimum width=\paperwidth, anchor=west, minimum height=1cm, opacity=0.8] at ([yshift=8cm]current page.south west) {};

      \node (student) at ([yshift=11cm,xshift=-1cm]current page.south east) {};
      \node[white,align=flush right,anchor=east] at (student.north east) {\coverpagefontstudent S. Tudent \\[2ex]\coverpagefontstudent 12345678\\[2ex]\coverpagefontstudent S.Tudent@student.hhs.nl\par};

      \node (opleiding) at ([yshift=6cm,xshift=-1cm]current page.south east) {};
      \node[white,align=flush right,anchor=east] at (opleiding.north east) {\coverpagefontopleiding Opleiding Civiele Techniek\par};

	  \node[white] (title) [align=center,text width=\paperwidth-8cm, anchor=north] at ([yshift=-3cm]current page.north) {\coverpagefonttitle\begin{hyphenrules}{nohyphenation} Watermanagement in droge omgevingen \end{hyphenrules}\par};

	  \node[white] (subtitle) [align=center,text width=\paperwidth-8cm, anchor=north] at ([yshift=-10cm]current page.north) {\coverpagefontisubtitle\begin{hyphenrules}{nohyphenation} --- Een innovatieve aanpak ---\end{hyphenrules}\par};
  \end{tikzpicture}

\frontmatter
\chapter{Voorwoord}
\kant[1-2]
{\bigskip\hfill\slshape \the\year, S. Tudent}
\tableofcontents
\chapter{Samenvatting}
\kant[1-2]

\mainmatter
\chapter{Inleiding}
\kant[1]
\section{More Kant}
\kant[1-6]

\chapter{Opdrachtsomschrijving}
\kant[7]
\section{More Kant}
\kant[8]

Zie figuur~\ref{fig:fig1}.

\begin{figure}[!ht]
\centering
\includegraphics[width=0.5\textwidth]{example-image-a}
\caption{This is an example of an image.}
\label{fig:fig1}
\end{figure}
\kant[9]

\chapter{Onderzoek}

\kant[1]
In~\eqref{equ:equ1} wordt aangegeven \ldots

\begin{equation}
\label{equ:equ1}
\int_{0}^{\infty} \frac{1}{x}\ln x\, \mathrm{d}x = \ldots
\end{equation}
\end{document}